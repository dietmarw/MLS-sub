\documentclass[../MLS.tex]{subfiles}
\begin{document}

\chapter{Synchronous Language Elements}\doublelabel{synchronous-language-elements}

This section presents language elements for describing synchronous
behavior suited for implementation of control systems.

\section{Introduction}\doublelabel{introduction2}

\subsection{Overview}\doublelabel{overview}

{[}\emph{This chapter defines additional kinds of discrete-time
variables and equations, as well as an additional kind of when-clause,
in order to define sampled data systems in a safe way, so that the
translator can provide good diagnostics in case of a modeling error. }

\emph{The following small example shows the most important elements}

\begin{figure}[H]
\caption{A continuous plant and a sampled data controller connected
together with sample and (zero-order) hold elements}
\begin{center}
\includegraphics[width=7in,height=4in]{plantmodel}
\end{center}
\end{figure}

\begin{itemize}
\item
  \emph{A periodic clock is defined with \textbf{Clock}(3). The argument
  of \textbf{Clock}(..) defines the sampling interval (for details see \autoref{clock-constructors}).}
\item
  \emph{Clocked variables (such as yd, xd, ud) are associated uniquely
  with a clock and can only be directly accessed when the associated
  clock is active. Since all variables in a clocked equation must belong
  to the same clock, clocking errors can be detected at compile time. If
  variables from different clocks shall be used in an equation, explicit
  cast operators must be used, such as \textbf{sample}(..) to convert
  from continuous-time to clocked discrete-time or \textbf{hold}(..) to
  convert from clocked discrete-time to continuous-time.}
\item
  \emph{A continuous-time variable is sampled at a clock tick with the
  \textbf{sample}(..) operator. The operator returns the value of the
  continuous-time variable when the clock is active.}
\item
  \emph{When no argument is defined for \textbf{Clock}(), the clock is
  deduced by clock inference.}
\item
  \emph{For a \textbf{when}-clause with an associated clock, all
  equations inside the \textbf{when}-clause are clocked with the given
  clock. All equations on an associated clock are treated together and
  in the same way regardless of whether they are inside a
  \textbf{when}-clause or not. This means that automatic sampling and
  hold of variables inside the \textbf{when}-clause does not apply
  (explicit sampling and hold is required) and that general equations
  can be used in such when-clauses (this is not allowed for
  \textbf{when}-clauses with Boolean conditions, that require a variable
  reference on the left-hand side of an equation).}
\item
  \emph{The \textbf{when}-clause in the controller could also be removed
  and the controller could just be defined by the equations:}

\begin{lstlisting}[language=modelica]
  // discrete controller
  E*xd = A*previous(xd) + B*yd;
  ud = C*previous(xd) + D*yd;
\end{lstlisting}
\item
  \emph{The operator \textbf{previous}(xd) returns the value of xd at
  the previous clock tick. At the first sample instant, the start value
  of xd is returned.}
\item
  \emph{A discrete-time signal (such as ud) is converted to a
  continuous-time signal with the \textbf{hold}(..) operator.}
\item
  \emph{If a variable belongs to a particular clock, then all other
  equations where this variable is used, with the exception of as
  argument to certain special operators, belong also to this clock, as
  well as all variables that are used in these equations. This property
  is used for ``clock inference'' and allows to define an associated
  clock only at a few places (above only in the sampler, whereas in the
  discrete controller and the hold the sampling period is inferred)}
\item
  \emph{The approach in this chapter is based on the clock calculus and
  inference system proposed by (Colaco and Pouzet 2003) and implemented
  in Lucid Synchrone version 2 and 3 (Pouzet 2006). However, the
  Modelica approach also uses multi-rate periodic clocks based on
  rational arithmetic introduced by (Forget et. al. 2008), as an
  extension of the Lucid Synchrone semantics. These approaches belong to
  the class of synchronous languages (Benveniste et. al. 2002). }
\end{itemize}

\subsection{Rationale for Clocked Semantics}\doublelabel{rationale-for-clocked-semantics}

\emph{Periodically sampled control systems could also be defined with
standard when-clauses, see \autoref{when-equations}, and the sample operator, see
\autoref{event-related-operators-with-function-syntax}. For example:}
\begin{lstlisting}[language=modelica]
when sample(0,3) then
  xd = A*pre(xd) + B*y;
  u = C*pre(xd) + D*y;
end when;
\end{lstlisting}

\emph{Equations in a when-clause with a Boolean condition have the
property that (a) variables on the left hand side of the equal sign are
assigned a value when the when-condition becomes true and otherwise hold
their value, (b) variables not assigned in the when-clause are directly
accessed (= automatic ``sample'' semantics), and (c) the variables
assigned in the when-clause can be directly accessed outside of the
when-clause (= automatic ``hold'' semantics). This approach to define
periodically sample data systems has the following drawbacks that are
not present with the solution in this chapter using clocks and clocked
equations:}

\begin{enumerate}
\item
  \emph{It is not possible to detect sampling errors due to the
  automatic sample and hold semantics. Examples: }

  \begin{enumerate}
  \def\labelenumii{\alph{enumii}.}
  \item
    \emph{If when-clauses in different blocks should belong to the same
    controller part, but by accident different when-conditions are
    given, then this is accepted (no error is detected)..}
  \item
    \emph{If a sampled data library such as the
    Modelica\_LinearSystems2.Contoller library is used, at every block
    the sampling of the block has to be defined as integer multiple of a
    base sampling rate. If several blocks should belong to the same
    controller part, and different integer multiples are given, then the
    translator has to accept this (no error is detected).}
  \end{enumerate}
\item
  \emph{Due to the automatic sample and hold semantics, all variables
  assigned in a when-clause of the above kind must have an initial value
  because they might be used, before they are assigned a value the first
  time. As a result, all these variables are ``discrete-time states''
  although in reality only a small subset of them need an initial
  value.}
\item
  \emph{Only a restricted form of equations can be used in a standard
  when-clause, since the left hand side has to be a variable, in order
  to identify the variables that are assigned in the when-clause. This
  is a severe restriction, especially if nonlinear control algorithms
  shall be defined. This restriction is not present for clocked
  equations.}
\item
  \emph{All equations belonging to a discrete controller must be in a
  when clause. If the controller is built-up with several building
  blocks, then the clock condition (sampling) must be explicitly
  propagated to all blocks. This is tedious and error prone. With
  clocked equations, the clock condition need to be defined only at one
  place, and otherwise is automatically propagated by clock inference.}
\item
  \emph{It is not possible to use a continuous-time model in when
  clauses (e.g. some advanced controllers use an inverse model of a
  plant in the feedforward path of the controller, see (Thümmel et. al.
  2005)). This powerful feature of Modelica to use a nonlinear plant
  model in a controller would require to export the continuous-time
  model with an embedded integration method and then import it in an
  environment where the rest of the controller is defined. With clocked
  equations, clocked controllers with continuous-time models can be
  directly defined in Modelica.}
\item
  \emph{At a sample instant, an event iteration occurs (as for any other
  event). A clocked partition, as well as a when-clause with a
  sample(..) is evaluated exactly once at such an event instant.
  However, the continuous-time model to which the sampled data
  controller is connected, will be evaluated several times when the
  overall system is simulated. With when-clauses, the continuous-time
  part is typically evaluated three times at a sample instant (once,
  when the sample instant is reached, once to evaluate the continuous
  equations at the sample instant, and once when an event iteration
  occurs since a discrete variable v is changed and \textbf{pre}(v)
  appears in the equations). With clocked equations, no event iteration
  is triggered if a clocked variable v is changed and
  \textbf{previous}(v) appears in the equations, because the event
  iteration cannot change the value of v. As a result, typically the
  simulation model is evaluated twice at a sample instant and therefore
  the simulation is more efficient with clocked equations.}
\end{enumerate}

{]}

\section{Definitions}\doublelabel{definitions}

In this section various terms are defined.

\subsection{Clocks and Clocked Variables}\doublelabel{clocks-and-clocked-variables}

In \autoref{discrete-time-expressions} the term ``discrete-time'' Modelica expression and in
\autoref{continuous-time-expressions} the term ``continuous-time'' Modelica expression is
defined. In this chapter, two additional kinds of discrete-time
expressions/variables are defined that are associated to clocks and are
therefore called ``clocked discrete-time'' expressions:

\begin{longtable}[]{|p{7.3cm}|p{7.3cm}|}
\hline
\multicolumn{2}{|p{14.6cm}|}{\textbf{The different kinds of discrete-time variables in Modelica}}\\ \hline
\endhead
\begin{tabular}{p{7cm}}
\includegraphics[width=3in,height=1.875in]{piecewise}
\end{tabular}&\begin{tabular}{p{7cm}}\textbf{Piecewise-constant variables (see \autoref{discrete-time-expressions})}

Variables \textbf{m}(t) of base type Real, Integer, Boolean,
enumeration, and String that are \underline{constant} inside each interval
t\textsubscript{i} $\le$ t \textless{} t\textsubscript{i+1} (= piecewise
constant continuous-time variables). In other words, \textbf{m}(t)
\underline{changes} value \underline{only at events}. This means, \textbf{m}(t) =
\textbf{m}(t\textsubscript{i}), for t\textsubscript{i} $\le$ t \textless{}
t\textsubscript{i+1}. Such variables depend continuously on time and
they are discrete-time variables.
\end{tabular}\\ \hline
\begin{tabular}{p{7cm}}
\includegraphics[width=3in,height=1.875in]{clock}
\end{tabular}&
\begin{tabular}{p{7cm}}\textbf{Clock variables}

Clock variables \textbf{c}(t\textsubscript{i}) are of base type Clock. A
clock is either defined by a constructor {[}\emph{such as Clock(3)}{]}
that defines when the clock ticks (is active) at a particular time
instant, or it is defined with clock operators relatively to other
clocks, see \autoref{base-clock-conversion-operators}.

{[}\emph{Examples:}
\begin{lstlisting}[language=modelica]
  Clock c1 = Clock(...);
  Clock c2 = c1;
  Clock c3 = subSample(c2,4);
\end{lstlisting}
{]}
\end{tabular}\\ \hline
\begin{tabular}{p{7cm}}
\includegraphics[width=3in,height=1.875in]{clocked}
\end{tabular}&
\begin{tabular}{p{7cm}}\textbf{Clocked variables}

The elements of clocked variables \textbf{r}(t\textsubscript{i}) are of
base type Real, Integer, Boolean, enumeration, String that are
associated uniquely with a clock \textbf{c}(t\textsubscript{i}). A
clocked variable can only be directly accessed at the event instant
where the associated clock is \textbf{active}. A constant and a
parameter can always be used at a place where a clocked variable is
required.

At time instants where the associated clock is not active, the value of
a clocked variable can be inquired by using an explicit cast operator,
see below. In such a case a ``hold'' semantics is used, in other words
the value of the clocked variable from the last event instant is used.
{[}\emph{This is visualized in the left figure with the dashed green
lines.}{]}\end{tabular}\\ \hline
\end{longtable}

{]}

\subsection{Base-Clock and Sub-Clock Partitions}\doublelabel{base-clock-and-sub-clock-partitions}

The following concepts are used:

\begin{itemize}
\item
  A ``\textbf{base-clock partition}'' identifies a set of equations and
  a set of variables which must be executed together in one task.
  Different base-clock partitions can be associated to separate tasks
  for asynchronous execution.
\item
  A ``\textbf{sub-clock partition}'' identifies a subset of equations
  and a subset of variables of a base-clock partition which are
  partially synchronized with other sub-clock partitions of the same
  base-clock partition, i.e., synchronized when the ticks of the
  respective clocks are simultaneous.
\end{itemize}

\subsection{Argument Restrictions (Component Expression)}\doublelabel{argument-restrictions-component-expression}

The built-in operators (with function syntax) defined in the following
sections have partially restrictions on their input arguments that are
not present for Modelica functions. To define the restrictions, the
following term is defined:

\begin{description}
\item{\textbf{Component Expression}:}

A Component Reference which is an Expression, i.e. does not refer to
models or blocks with equations. It is an instance of a (a) base type,
(b) derived type, (c) record, (d) an array of such an instance (a-c),
(e) one or more elements of such an array (d) defined by index
expressions which are parameter expressions (see below), or (f) an
element of records. {[}\emph{The essential features are that one or
several values are associated with the instance, that start values can
be defined on these values, and that no equations are associated with
the instance. A Component Expression can be constant or can vary with
time.}{]}
\end{description}

In the following sections the following notation is partially used when
defining the operators:

% This should ideally be a set of definitions instead
\begin{itemize}
\item
  \textbf{The input argument is a Component Expression}:

  The meaning is that the input argument when calling the operator must
  be a Component Expression.

  {[}\emph{The reason for this restriction is that the start value of
  the input argument is returned before the first tick of the clock of
  the input argument and this is not possible for a general
  expression.}

  \emph{Examples:}
\begin{lstlisting}[language=modelica]
  Real u1;
  Real u2[4];
  Complex c;
  Resistor R;
  ...
  y1 = previous(u1);    // fine
  y2 = previous(u2);    // fine
  y3 = previous(u2[2]); // fine
  y4 = previous(c.im);  // fine
  y5 = previous(2*u);   // error (general expression, no Component Expression)
  y6 = previous(R);     // error (component, no Component Expression)
\end{lstlisting}
{]}
\item
  \textbf{The input argument is a parameter expression}:

  The meaning is that the input argument when calling the operator must
  have parameter variability, that is the argument must depend directly
  or indirectly only on parameters, constants or literals, see 
  \autoref{variability-of-expressions}.
  
  {[}\emph{The reason for this restriction is that the value of the
  input argument needs to be evaluated during translation, in order that
  clock analysis can be performed during translation.}

  \emph{Examples:}
\begin{lstlisting}[language=modelica]
  Real u;
  parameter Real p=3;
  ...
  y1 = subSample(u, factor=3);       // fine (literal)
  y2 = subSample(u, factor=2*p - 3); // fine (parameter expression)
  y3 = subSample(u, factor=3*u);     // error (general expression)
\end{lstlisting}
{]}
\item
  \textbf{The input argument is an expression:}

  There is no restriction on the input argument when calling the
  operator. This notation is used to emphasis when a standard function
  call is used (``is an expression''), instead of restricting the input
  (``is a Component Expression'').
\end{itemize}

\section{Clock Constructors}\doublelabel{clock-constructors}

The following overloaded constructors are available to generate clocks:

\begin{longtable}[]{|p{3cm}|p{12cm}|}
\hline \endhead
\textbf{Clock}()
&
\begin{tabular}{@{}p{119mm}@{}}
\textbf{Inferred Clock}\\

The operator returns a clock that is inferred.

{[}\emph{Example}:
\begin{lstlisting}[language=modelica]
when Clock() then  // equations are on the same clock
  x = A*previous(x) + B*u;
  Modelica.Utilities.Streams.print
    ("clock ticks at = " + String(sample(time)));
end when;
\end{lstlisting}
\emph{Note, in most cases, the operator is not needed and equations
could be written without a when-clause (but not in the example above,
since the ``print'' statement is otherwise not associated to a clock).
This style is useful if a modeler would clearly like to mark the
equations that must belong to one clock (although a tool could figure
this out as well, if the when-clause is not present).}{]}
\end{tabular}\\ \hline
\begin{tabular}{@{}p{29mm}@{}}
\textbf{Clock}(\newline
 intervalCounter,\newline
 resolution)
\end{tabular}
&
\begin{tabular}{@{}p{119mm}@{}}
\textbf{Clock with Rational Interval}\\

The first input argument, intervalCounter, is a clocked Component
Expression (see \autoref{argument-restrictions-component-expression}) or a parameter expression of type
\textbf{Integer} with min=0. The optional second argument resolution
(default=1) is a parameter expression of type Integer with min=1 and
unit ``Hz''. If intervalCounter is a parameter expression with value
zero, the period of the clock is derived by clock inference, see 
\autoref{sub-clock-inferencing}. The output argument is of base type Clock that ticks when time
becomes t\textsubscript{start}, t\textsubscript{start}+interval1,
t\textsubscript{start}+interval1+interval2, ... The clock starts at the
start of the simulation t\textsubscript{start} or when the controller is
switched on. At the start of the simulation, previous(intervalCounter) =
intervalCounter.start and the clocks ticks the first time. At the first
clock tick intervalCounter must be computed and the second clock tick is
then triggered at interval1=intervalCounter/resolution. At the second
clock tick at time tstart+interval1, a new value for intervalCounter
must be computed and the next clock tick is scheduled at interval2 =
intervalCounter/resolution, and so on. If interval is a parameter
expression, the clock defines a periodic clock.

{[}\emph{The given interval and time shift can be modified by using the
subSample, superSample, shiftSample and backSample operators on the
returned clock, see \autoref{sub-clock-conversion-operators}.}

\emph{Example:}
\begin{lstlisting}[language=modelica]
  // first clock tick: previous(nextInterval)=2
  Integer nextInterval(start=2);
  Real y1(start=0);
  Real y2(start=0);
equation
  when Clock(2,1000) then
    // periodic clock that ticks at 0, 0.002, 0.004, ...
    y1 = previous(y1) + 1;
  end when;

  when Clock(nextInterval, 1000) then
    // interval clock that ticks at 0, 0.003, 0.007, 0.012, ...
    nextInterval = previous(nextInterval) + 1;
    y2 = previous(y2) + 1;
  end when;
\end{lstlisting}
{]}

Note that operator interval(c) of Clock c =
Clock(nextInterval,resolution) returns:\newline
previous(intervalCounter)/resolution; // in seconds
\end{tabular}\\ \hline
\textbf{Clock}(interval)
&
\begin{tabular}{@{}p{119mm}@{}}
\textbf{Clock with Real Interval}\\

The input argument, interval, is a clocked Component Expression (see
\autoref{argument-restrictions-component-expression}) or a parameter expression of type \textbf{Real} with
min=0.0 and unit ``s''. The output argument is of base type Clock that
ticks when time becomes t\textsubscript{start},
t\textsubscript{start}+interval1,
t\textsubscript{start}+interval1+interval2, ... The clock starts at the
start of the simulation t\textsubscript{start} or when the controller is
switched on. Here the next clock tick is scheduled at interval1 =
\textbf{previous}(interval) = interval.start. At the second clock tick
at time t\textsubscript{start}+interval1, the next clock tick is
scheduled at interval2 = \textbf{previous}(interval), and so on. If
interval is a parameter expression, the clock defines a periodic clock.

{[}\emph{Note, the clock is defined with \textbf{previous}(interval).
Therefore, for sorting the input argument is treated as known.}
\emph{The given interval and time shift can be modified by using the
subSample, superSample, shiftSample and backSample operators on the
returned clock, see \autoref{sub-clock-conversion-operators}. There are restrictions where
this operator can be used, see Clock expressions below.}{]}
\end{tabular}\\ \hline
\begin{tabular}{@{}p{29mm}@{}}
\textbf{Clock}(\newline
  condition,\newline
  startInterval)
\end{tabular}
&
\begin{tabular}{@{}p{119mm}@{}}
\textbf{Clock with Boolean Condition}\\

The input argument, condition, is a continuous-time expression of type
Boolean. The optional startInterval argument (default = 0.0) is the
value returned by the operator \textbf{interval}() at the first tick of
the clock, see \autoref{initialization-of-clocked-partitions}. The output argument is of base type Clock
that ticks when \textbf{edge}(condition) becomes true.

{[}\emph{This clock is used to trigger a clocked partition due to a
state event, that is a zero-crossing of a Real variable, in a
continuous-time partition or due to a hardware interrupt that is modeled
as Boolean in the simulation model. Example:}

\begin{lstlisting}[language=modelica]
  Clock c = Clock(angle > 0, 0.1) // before first tick of c:
                                  // interval(c) = 0.1
\end{lstlisting}

\emph{The implicitly given interval and time shift can be modified by
using the subsample, superSample, shiftSample and backSample operators
on the returned clock, see \autoref{sub-clock-conversion-operators}, provided the base
interval is not smaller than the implicitly given interval.} {]}
\end{tabular}\\ \hline
\begin{tabular}{@{}p{29mm}@{}}
\textbf{Clock}(\newline
 c,\newline
 solverMethod)\newline
\end{tabular}
&
\begin{tabular}{@{}p{119mm}@{}}
\textbf{Solver Clock}\\

The first input argument ``c'' is a clock and the operator returns this
clock. The returned clock is associated with the second input argument
of type String ``solverMethod''. The meaning of solverMethod is defined
in \autoref{solver-methods}. If the second input argument solverMethod is an empty
String, then no integrator is associated with the returned clock.

{[}\emph{Examples:}
\begin{lstlisting}[language=modelica]
  Clock c1 = Clock(1,10) // 100 ms, no solver
  Clock c2 = Clock(c1, "ImplicitTrapezoid");
    // 100 ms, ImplicitTrapezoid solver
  Clock c3 = Clock(c2, ""); // 100 ms, no solver
\end{lstlisting}
{]}\strut
\end{tabular}
\\ \hline

\end{longtable}

Besides inferred clocks and solver clocks, one of the following mutually
exclusive associations of clocks are possible in one base partition:

\begin{enumerate}
\item
  One or more Rational interval clocks, provided they are consistent
  with each other, see \autoref{sub-clock-inferencing}.\\
  {[}\emph{For example, assume ``y = subSample(u)'', and Clock(1,10) is
  associated to ``u'' and Clock(2,10) is associated with ``y'', then
  this is correct, but it would be an error if ``y'' is associated to a
  Clock (1,3).} {]}
\item
  Exactly one Real interval clock. {[}\emph{Assume``Clock c =
  Clock(2.5)'', then variables in the same base partition can be
  associated multiple times with ``c'' but not multiple times with
  ``Clock(2.5)''}{]}
\item
  Exactly one Boolean clock.
\item
  A default clock, if neither a Real interval, nor a Rational interval
  nor a Boolean clock is associated with a base partition. In this case
  the default clock is associated with the fastest sub-clock partition.
  {[}\emph{Typically, a tool will use Clock(1.0) as a default clock and
  will raise a warning, that it selected a default clock.}{]}
\end{enumerate}

Clock variables can be used in a restricted form of expressions.
Generally, every expression containing clock variables must have
parametric variability {[}\emph{in order that clock analysis can be
performed when translating a model.}{]}. Otherwise, the following
expressions are allowed:

\begin{itemize}
\item
  Declaring arrays of clocks {[}\emph{Example:} \lstinline!Clock c1[3] ={Clock(1), Clock(2), Clock(3)}! {]}
% Ok, that bug in pdflatex was weird: \emph{Example: \lstinline!Clock c1[3] ={Clock(1), Clock(2), Clock(3)}!} {]}
\item
  Array constructors of clocks: \lstinline!{}, [], cat(...)!.
\item
  Array access of clocks {[}\emph{Example:  \lstinline!sample(u, c1[2])!}{]}
\item
  Equality of clocks {[}\emph{Example: \lstinline!c1 = c2!}{]}.
\item
  If-expressions of clocks in equations\\
  {[}\emph{Example: \lstinline!Clock c2 = if f>0 then subSample(c1, f) elseif f<0 then superSample(c1, f) else c1!}{]}.
\item
  Clock variables can be declared in models, blocks, connectors, and
  records,. A Clock variable can be declared with the prefixes
  \textbf{input}, \textbf{output}, \textbf{inner}, \textbf{outer}, but
  \textbf{not} with the prefixes \textbf{flow}, \textbf{stream},
  \textbf{discrete}, \textbf{parameter}, or \textbf{constant}
  {[}\emph{Example: \textbf{connector} ClockInput = \textbf{input}
  Clock;}{]}
\end{itemize}

\section{Discrete States}\doublelabel{discrete-states}

The previous value of a clocked variable can be accessed with the
previous operator. Such a variable is called a clocked state variable.

\begin{longtable}[]{|l|p{12cm}|}
\hline \endhead
\textbf{previous}(u) & The input argument is a Component Expression (see
\autoref{argument-restrictions-component-expression}) or a parameter expression. The return argument has the
same type as the input argument. Input and return arguments are on the
same clock. At the first tick of the clock of u or after a reset
transition (see \autoref{reset-handling}), the start value of u is returned, see
\autoref{initialization-of-clocked-partitions}. At subsequent activations of the clock of u, the value of
u from the previous clock activation is returned.\\ \hline
\end{longtable}

\section{Partitioning Operators}\doublelabel{partitioning-operators}

A set of ``clock conversion operators'' together act as boundaries
between different clock partitions.

\subsection{Base-clock conversion operators}\doublelabel{base-clock-conversion-operators}

The following operators convert between a continuous-time and a
clocked-time representation and vice versa:

\begin{longtable}[]{|l|p{12cm}|}
\hline \endhead
\textbf{sample}(u, c) &
Input argument u is a continuous-time expression according to 
\autoref{continuous-time-expressions}. The optional input argument c is of type Clock. The operator
returns a clocked variable that has c as associated clock and has the
value of the left limit of u when c is active (that is the value of u
just before the event of c is triggered). If argument c is not provided,
it is inferred, see \autoref{sub-clock-inferencing}.

{[}\emph{Since the operator returns the left limit of u, it introduces
an infinitesimal small delay between the continuous-time and the clocked
partition. This corresponds to the reality, where a sampled data system
cannot act infinitely fast and even for a very idealized simulation, an
infinitesimal small delay is present. The consequences for the sorting
are discussed below.}

\emph{Input argument u can be a general expression, because the argument
is continuous-time and therefore has always a value. It can also be a
constant, a parameter or a piecewise constant expression. }

\emph{Note that \textbf{sample}() is an overloaded function: If
\textbf{sample}(..) has two input arguments and the second argument is
of type Real, it is the operator from \autoref{event-related-operators-with-function-syntax}. If
\textbf{sample}(..) has one input argument, or it has two input
arguments and the second argument if of type Clock, it is the base-clock
conversion operator from this section.}{]}\\ \hline
\textbf{hold}(u) &
Input argument u is a clocked Component Expression (see \autoref{argument-restrictions-component-expression})
or a parameter expression. The operator returns a piecewise constant
signal of the same type of u. When the clock of u ticks, the operator
returns u and otherwise returns the value of u from the last clock
activation. Before the first clock activation of u, the operator returns
the start value of u, see \autoref{initialization-of-clocked-partitions}.

{[}\emph{Since the input argument is not defined before the first tick
of the clock of u, the restriction is present, that it must be a
Component Expression (or a parameter expression), in order that the
initial value of u can be used in such a case.}{]}\\ \hline
\end{longtable}

{[}\emph{Example: }

\emph{Assume there is the following model:}

\begin{lstlisting}[language=modelica]
  Real y(start=1), yc;
equation
  der(y) + y = 2;
  yc = sample (y, Clock(0.1));
initial equation
  der(y) = 0;
\end{lstlisting}
\emph{The value of yc at the first clock tick is yc=2 (and not yc=1).
The reason is that the continuous-time model der(y)+y=2 is first
initialized and after initialization y has the value 2. At the first
clock tick at time=0, the left limit of y is 2 and therefore yc = 2.}

\emph{Sorting of a simulation model:\\
Since sample(u) returns the left limit of u, and the left limit of u is
a known value, all inputs to a base-clock partition are treated as known
during sorting. Since a periodic and interval clock can tick at most
once at a time instant, and since the left limit of a variable does not
change during event iteration (i.e., re-evaluating a base-clock
partition associated with a condition clock always gives the same result
because the sample(u) inputs do not change and therefore need not to be
re-evaluated) all base-clock partitions, see \autoref{base-clock-partitioning}, need
not to be sorted with respect to each other. Instead, at an event
instant, active base-clock partitions can be evaluated first (and once)
in any order. Afterwards, the continuous-time partition is evaluated.
Event iteration takes place only over the continuous-time partition. In
such a scenario, accessing the left limit of u in sample(u) just means
to pick the latest available value of u when the partition is entered,
storing it in a local variable of the partition and only using this
local copy during evaluation of the equations in this partition.}{]}

\subsection{Sub-clock conversion operators}\doublelabel{sub-clock-conversion-operators}

The following operators convert between synchronous clocks:

\begin{longtable}[]{|p{4cm}|p{11cm}|}
\hline \endhead
\multicolumn{2}{|p{15cm}|}{
The operators in this table have the following properties:

The input argument u is a clocked expression or an expression of type
Clock. {[}\emph{The operators can operate on all types of clocks}{]}. If
u is a clocked expression, the operator returns a clocked variable that
has the same type as the expression. If u is an expression of type
Clock, the operator returns a Clock.

The optional input arguments factor (default=0, min=0), and resolution
(default=1, min=1) are parameter expressions of type Integer.

The input arguments shiftCounter and backCounter are parameter
expressions of type Integer (min=0).}
\\ \hline
\textbf{subSample}(u, factor)
&
The clock of y = \textbf{subSample}(u,factor) is factor-times slower
than the clock of u. At every factor ticks of the clock of u, the
operator returns the value of u.. The first activation of the clock of y
coincides with the first activation of the clock of u. If argument
factor is not provided or is equal to zero, it is inferred, see 
\autoref{sub-clock-inferencing}.
\\ \hline
\textbf{superSample}(u, factor)
&
The clock of y = \textbf{superSample}(u,factor) is factor-times faster
than the clock of u. At every tick of the clock of y, the operator
returns the value of u from the last tick of the clock of u. The first
activation of the clock of y coincides with the first activation of the
clock of u. If argument factor is not provided or is equal to zero, it
is inferred, see \autoref{sub-clock-inferencing}. If a Boolean clock is associated to a
base-clock partition, all its sub-clock partitions must have resulting
clocks that are sub-sampled with an Integer factor with respect to this
base clock.
{[}\emph{Example:}
\begin{lstlisting}[language=modelica]
  Clock u = Clock(x > 0);
  Clock y1 = subSample(u,4);
  Clock y2 = superSample(y1,2); // fine; y2 =  subSample(u,2)
  Clock y3 = superSample(u ,2); // error
  Clock y4 = superSample(y1,5); // error
\end{lstlisting}
  {]}\\ \hline
\begin{tabular}{@{}p{4cm}@{}}
\textbf{shiftSample}(u,\\
shiftCounter, resolution)
\end{tabular}
& {[}\emph{The first activation of the clock of y =
\textbf{shiftSample}(..) is shifted in time
shiftCounter/resolution*interval(u) later than the first activation of
the clock of u.}{]}.

Conceptually, the operator constructs a clock ``cBase''
\begin{lstlisting}[language=modelica]
Clock cBase = subSample(superSample(u,resolution), shiftCounter)
\end{lstlisting}
and the clock of y = \textbf{shiftSample}(..) starts at the second clock
tick of cBase. At every tick of the clock of y, the operator returns the
value of u from the last tick of the clock of u.

{[}\emph{Note, due to the restriction of superSample on Boolean clocks,
shiftSample can only shift the number of ticks of the Boolean clock, but
cannot introduce new ticks. Example:}
\begin{lstlisting}[language=modelica]
// Rational interval clock
Clock u  = Clock(3, 10); // ticks: 0, 3/10, 6/10, ..
Clock y1 = shiftSample(u,1,3); // ticks: 1/10, 4/10,
...
// Boolean clock
Clock u = Clock(sin(2*pi*time)>0, startInterval=0.0)
// ticks: 0.0, 1.0, 2.0, 3.0, ...
Clock y1 = shiftSample(u,2); // ticks: 2.0, 3.0, ...
Clock y2 = shiftSample(u,2,3);// error (resolution must be 1)
\end{lstlisting}
{]}\\ \hline
\begin{tabular}{@{}p{4cm}@{}}
\textbf{backSample}(u,\\
backCounter, resolution)
\end{tabular}
&
The input argument u is either a Component Expression (see 
\autoref{argument-restrictions-component-expression}) or an expression of type Clock. {[}\emph{The first activation of
the clock of y = \textbf{backSample}(..) is shifted in time
backCounter/resolution*\textbf{interval}(u) before the first activation
of the clock of u}{]}. Conceptually, the operator constructs a clock
``cBase''
\begin{lstlisting}[language=modelica]
Clock cBase = subSample(superSample(u,resolution), backCounter)
\end{lstlisting}
and the clock of y = \textbf{shiftSample}(..) is shifted a time duration
before the clock of u, such that this duration is identical to the
duration between the first and second clock tick of cBase. It is an
error, if the clock of y starts before the base clock of u. At every
tick of the clock of y, the operator returns the value of u from the
last tick of the clock of u. If u is a clocked Component Expression, the
operator returns the start value of u, see \autoref{initialization-of-clocked-partitions}, before the
first tick of the clock of u.

\emph{{[}Example:}
\begin{lstlisting}[language=modelica]
// Rational interval clock 1

Clock u  = Clock(3, 10); // ticks: 0, 3/10, 6/10, ..
Clock y1 = shiftSample(u,3); // ticks: 9/10, 12/10, ..
Clock y2 = backSample(y1,2); // ticks: 3/10, 6/10,
...
Clock y3 = backSample(y1,4); // error (ticks before u)
Clock y4 = shiftSample(u,2,3); // ticks: 2/10, 5/10,
...
Clock y5 = backSample(y4,1,3); // ticks: 1/10, 4/10,
...
// Boolean clock
Clock u = Clock(sin(2*pi*time) > 0, startInterval=xx)
// ticks: 0, 1.0, 2.0, 3.0, ....
Clock y1 = shiftSample(u,3); // ticks: 3.0, 4.0, ...
Clock y2 = backSample(y1,2); // ticks: 1.0, 2.0, ...
\end{lstlisting}
{]}
\\ \hline
\textbf{noClock}(u)
&
The clock of y = \textbf{noClock}(u) is always inferred. At every tick
of the clock of y, the operator returns the value of u from the last
tick of the clock of u. If \textbf{noClock}(u) is called before the
first tick of the clock of u, the start value of u is returned.\\ \hline
\end{longtable}

\emph{{[}Clarification of backSample(..) operator:}

\emph{Let a and b be positive integers with a \textless{} b, and}
\begin{lstlisting}[language=modelica]
yb = backSample (u, a , b)
ys = shiftSample(u, b-a, b)
\end{lstlisting}
\emph{Then when ys exists, also yb exists and ys = yb.\\
The variable yb exists for the above parameterization with a \textless{}
b one clock tick before ys. Therefore, \textbf{backSample} is basically
a \textbf{shiftSample} with a different parameterization and the clock
of \textbf{backSample.y} ticks before the clock of u. Before the clock
of u ticks, yb = u.start.}

\emph{Clarification of noClock(..) operator:}

\emph{Note, that noClock(u) is not equivalent to sample(hold(u)).
Consider the following model:}

\begin{lstlisting}[language=modelica]
model NoClockVsSampleHold
  Clock clk1 = Clock(0.1);
  Clock clk2 = subSample(clk1,2);
  Real x(start=0), y(start=0), z(start=0);
equation
  when clk1 then
    x = previous (x) + 0.1;
  end when;
  when clk2 then
    y = noClock (x); // most recent value of x
    z = sample (hold(x)); // left limit of x (infinitesimally delayed)!
  end when;
end NoClockVsSampleHold;
\end{lstlisting}

\emph{Due to the infinitesimal delay of sample; z will not show the
current value of x as clk2 ticks, but will show its previous value (left
limit). However, y will show the current value, since it has no
infinitesimal delay.{]}}

\section{Clocked When Clause}\doublelabel{clocked-when-clause}

In addition to the previously discussed conditional when-clause, a
\emph{clocked} when-clause is introduced:
\begin{lstlisting}[language=modelica,escapechar=!]
when !\emph{clock-expression}! then
  !\emph{clocked-equation}!
  ...
end when;
\end{lstlisting}

The clocked when-clause cannot be nested and does not have any elsewhen
part. It cannot be used inside an algorithm. General equations are
allowed in a clocked when-clause.

For a clocked when-clause, all equations inside the when-clause are
clocked with the same clock given by the \emph{clock-expression}.

\section{Clock Partitioning}\doublelabel{clock-partitioning}

This section defines how clock-partitions and clocks associated with
equations are inferred. {[}\emph{Typically clock partitioning is
performed before sorting the equations. The benefit is that clocking and
symbolic transformation errors are separated.}{]}

Every clocked variable is uniquely associated with exactly one clock.

After model flattening, every equation in an equation section, every
expression and every algorithm section is either continuous-time, or it
is uniquely associated with exactly one clock. In the latter case it is
called a clocked equation, a clocked expression or clocked algorithm
section respectively. The associated clock is either explicitly defined
by a when-clause, see \autoref{sub-clock-conversion-operators}, or it is implicitly defined by the
requirement that a clocked equation, a clocked expression and a clocked
algorithm section must have the same clock as the variables used in them
with exception of the expressions used as first arguments in the
conversion operators of \autoref{partitioning-operators}. Clock inference means to infer the
clock of a variable, an equation, an expression or an algorithm section
if the clock is not explicitly defined and is deduced from the required
properties in the previous two paragraphs.

All variables in an expression without clock conversion operators must
have the same clock to infer the clocks for each variable and
expression. The clock inference works both forward and backwards
regarding the data flow and is also being able to handle algebraic
loops. The clock inference method uses the set of variable incidences of
the equations, i.e., what variables that appear in each equation.

Note that incidences of the first argument of clock conversion operators
of \autoref{partitioning-operators} are handled specially.

\subsection{Flattening of Model}\doublelabel{flattening-of-model}

The clock partitioning is conceptually performed after model flattening,
i.e., redeclarations have been elaborated, arrays of model components
expanded into scalar model components, and overloading resolved.
Furthermore, function calls to inline functions have been inlined.
{[}\emph{This is called ``conceptually'', because a tool might do this
more efficiently in a different way, provided the result is the same as
if everything is flattened. For example, array and matrix equations and
records don't not need to be expanded if they have the same clock.}{]}

Furthermore, each non-trivial expression (non-literal, non-constant,
non-parameter, non-variable), expr\textsubscript{i}, appearing as first
argument of any clock conversion operator is recursively replaced by a
unique variable, v\textsubscript{i}, and the equation v\textsubscript{i}
= expr\textsubscript{i} is added to the equation set.

\subsection{Connected Components of the Equations and Variables Graph}\doublelabel{connected-components-of-the-equations-and-variables-graph}

Consider the set E of equations and the set V of unknown variables (not
constants and parameters) in a flattened model, i.e. M = \textless{}E,
V\textgreater{}. The partitioning is described in terms of an undirected
graph \textless{}N,~F\textgreater{} with the nodes N being the set of
equations and variables, N = E + V. The set incidence(e) for an equation
e in E is a subset of V, in general, the unknowns which lexically appear
in e. There is an edge in F of the graph between an equation, e, and a
variable, v, if v = incidence(e):

$$F = \{(e, v) : e \in E , v \in \text{incidence}(e)\}$$

A set of clock partitions is the ``connected components'' (Wikipedia,
``Connected components'') of this graph with appropriate definition of
the incidence operator.

\subsection{Base-clock Partitioning}\doublelabel{base-clock-partitioning}

The goal is to identify all clocked equations and variables that should
be executed together in the same task, as well as to identify the
continuous-time partition.

The base-clock partitioning is performed with base-clock inference which
uses the following incidence definition:

% This is not how you are supposed to format things in LaTeX
\begin{tabular}{p{1cm}p{1cm}p{1cm}p{12cm}}
\multicolumn{4}{p{15cm}}{\textbf{incidence}(e) = the \emph{unknown} variables, as well as
variables x in \textbf{der}(x), \textbf{pre}(x), and \textbf{previous}(x),}\\
&\multicolumn{3}{p{14cm}}{which lexically appear in e}\\
&&\multicolumn{2}{p{13cm}}{except as first argument of base-clock conversion operators: sample() and hold().}
\end{tabular}

The resulting set of connected components, is the partitioning of the
equations and variables, B\textsubscript{i} =
\textless{}E\textsubscript{i}, V\textsubscript{i}\textgreater{},
according to base-clocks and continuous-time partitions.

The base clock partitions are identified as \textbf{clocked} or as
\textbf{continuous-time partitions} according to the following
properties:

A variable u in \textbf{sample}(u) and a variable y in y =
\textbf{hold}(ud) is in a continuous-time partition.

Correspondingly, variables u and y in y = \textbf{sample}(uc), y =
\textbf{subSample}(u), y = \textbf{superSample}(u), y =
\textbf{shiftSample}(u), y = \textbf{backSample}(u), y =
\textbf{previous}(u), are in a clocked partition. Equations in a clocked
when clause are also in a clocked partition.
Other partitions where none of the variables in the partition are
associated with any of the operators above have an unspecified partition
kind and are considered continuous-time partitions.

All continuous-time partitions are collected together and form ``the''
continuous-time partition.

{[}\emph{Example:}
\begin{lstlisting}[language=modelica]
  // Controller 1
  ud1 = sample(y,c1);
  0 = f1(yd1, ud1, previous(yd1));

  // Controller 2
  ud2 = superSample(yd1,2);
  0 = f2(yd2, ud2);

  // Continuous-time system
  u = hold(yd2);
  0 = f3(der(x1), x1, u);
  0 = f4(der(x2), x2, x1);
  0 = f5(der(x3), x3);
  0 = f6(y, x1, u);
\end{lstlisting}

\emph{After base clock partitioning, the following partitions are
identified:}

\begin{lstlisting}[language=modelica]
  // Base partition 1 // clocked partition
  ud1 = sample (y,c1); // incidence(e) = {ud1}
  0 = f1(yd1, ud1, previous(ud1)); // incidence(e) = {yd1,ud1}
  ud2 = superSample (yd1,2); // incidence(e) = {ud2, yd1}
  0 = f2(yd2, ud2); // incidence(e) = {yd2, ud2}

  // Base partition 2 // continuous-time partition
  u = hold (yd2); // incidence(e) = {u}
  0 = f3(der(x1), x1, u); // incidence(e) = {x1,u}
  0 = f4(der(x2), x2, x1); // incidence(e) = {x2,x1}
  0 = f6(y, x1, u); // incidence(e) = {y,x1,u}

  // Identified as separate partition, but belonging to partition 2
  0 = f5(der(x3), x3); // incidence(e) = {x3}
\end{lstlisting}
{]}

\subsection{Sub-clock Partitioning}\doublelabel{sub-clock-partitioning}

For each clocked partition B\textsubscript{i}, identified in 
\autoref{base-clock-partitioning}, the sub-clock partitioning is performed with sub-clock inference
which uses the following incidence definition:

% This is not how you are supposed to format things in LaTeX
\begin{tabular}{p{1cm}p{1cm}p{1cm}p{12cm}}
\multicolumn{4}{p{15cm}}{\textbf{incidence}(e) = the \emph{unknown} variables, as well as
variables x in \textbf{der}(x), \textbf{pre}(x), and
\textbf{previous}(x),}\\
&\multicolumn{3}{p{14cm}}{which lexically appear in e}\\
&&\multicolumn{2}{p{13cm}}{except as first argument of sub-clock conversion operators:}\\
&&&\multicolumn{1}{p{12cm}}{subSample, superSample, shiftSample, backSample, and noClock.}
\end{tabular}
The resulting set of connected components, is the partitioning of the
equations and variables, S\textsubscript{ij} =
\textless{}E\textsubscript{ij}, V\textsubscript{ij}\textgreater{},
according to sub-clocks.

It can be noted that:

$E_{ij} \bigcap E_{kl} = \emptyset~ \forall i\ne{}k, j\ne{}l$

$ V_{ij} \bigcap V_{kl} = \emptyset~ \forall i\ne{}k, j\ne{}l$

$V = \bigcup V_{ij}$

$E = \bigcup E_{ij}$

{[}\emph{Example:}

\emph{After sub-clock partitioning of the example from \autoref{base-clock-partitioning}, the following partitions are identified:}
\begin{lstlisting}[language=modelica]
  // Base partition 1 (clocked partition)
  // Sub-clock partition 1.1
  ud1 = sample (y,c1); // incidence(e) = {ud1}
  0 = f1(yd1,ud1,previous(yd1)); // incidence(e) = {yd1,ud1}

  // Sub-Clock partition 1.2
  ud2 = superSample (yd1,2); // incidence(e) = {ud2}
  0 = f2(yd2,ud2); // incidence(e) = {yd2,ud2}

  // Base partition 2 (no sub-clock partitioning, since continuous-time)
  u = hold (yd2);
  0 = f3(der(x1), x1, u);
  0 = f4(der(x2), x2, x1);
  0 = f5(der(x3), x3);
  0 = f6(y, x1, u);
\end{lstlisting}
{]}

\subsection{Sub-clock Inferencing}\doublelabel{sub-clock-inferencing}

For each base-clock partition, the base interval needs to be determined
and for each sub-clock partition, the sub-sampling factors and shift
need to be determined. For each sub-clock partition, the interval might
be rational or Real type and known or parametric or being unspecified.
The sub-clock partition intervals are constrained by subSample and
superSample factors which might be known (or parametric) or unspecified
and by shiftSample shiftCounter and resolution or backSample,
backCounter and resolution. This constraint set is used to solve for all
intervals and sub-sampling factors and shift of the sub-clock
partitions. The model is erroneous if no solution exist.

{[}\emph{It must be possible to determine that the constraint set is
valid at compile time. However, in certain cases, it could be possible
to defer providing actual numbers until run-time.} {]}

It is required that accumulated sub- and super sampling factors in the
range of 1 to 2\textsuperscript{63} can be handled.

{[}\emph{64 bit internal representation of numerator and denominator
with sign can be used and gives\\
minimum resolution 1.08E-19 seconds and maximum range 9.22E+18 seconds =
2.92E+11 years.}{]}

\section{Continuous-Time Equations in Clocked Partitions}\doublelabel{continuous-time-equations-in-clocked-partitions}

{[}\emph{The goal is that every continuous-time Modelica model can be
utilized in a sampled data control system. This is achieved by solving
the continuous-time equations with a defined integration method between
clock ticks. With this feature, it is for example possible to invert the
nonlinear dynamic model of a plant, see (Thümmel et.al. 2005), and use
it in a feedforward path of an advanced control system that is
associated with a clock.}

\emph{This feature also allows to define multi-rate systems: Different
parts of the continuous-time model are associated to different clocks
and are solved with different integration methods between clock ticks,
e.g., a very fast sub-system with an implicit solver with a small
step-size and a slow sub-system with an explicit solver with a large
step-size.}{]}

With the language elements defined in this section, continuous-time
equations can be used in clocked partitions. Hereby, the continuous-time
equations are solved with the defined integration method between clock
ticks.

From the \textbf{view of the continuous-time partition}, the clock ticks
are not interpreted as events, but as step-sizes of the integrator that
the integrator must exactly hit. {[}\emph{This is the same assumption as
for manually discretized controllers, such as the z-transform.}{]} So no
event handling is triggered at clock ticks (provided an explicit event
is not triggered from the model at this time instant). {[}\emph{It is
not defined, how events are handled that appear when solving the
continuous-time partition. For example, a tool could handle events
exactly in the same way as for a usual simulation. Alternatively,
relations might be interpreted literally, so that events are no longer
triggered (in order that the time for an integration step is always the
same, as needed for hard real-time requirements).}{]}

From the \textbf{view of the clocked partition}, the continuous-time
partition is discretized and the discretized continuous-time variables
have only a value at a clock tick. Therefore, such a partition is
handled in the same way as any other clocked partition. Especially,
operators such as sample, hold, subSample must be used to communicate
signals of the discretized continuous-time partition with other
partitions. Hereby, a discretized continuous-time partition is seen as a
clocked partition.

\subsection{Clocked Discrete-Time and Clocked Discretized Continuous-Time Partition}\doublelabel{clocked-discrete-time-and-clocked-discretized-continuous-time-partition}

Additionally to the variability of expressions defined in \autoref{variability-of-expressions},
an orthogonal concept ``clocked variability'' is defined in this
section. If not explicitly stated otherwise, an expression with a
variability such as ``continuous-time'' or ``discrete-time'' means that
the expression is inside a partition that is not associated to a clock.
If an expression is present in a partition that is not a continuous-time
partition, it is a ``\textbf{clocked expression}'' and has
``\textbf{clocked variability}''.

After sub-clock inferencing, see \autoref{sub-clock-inferencing}, every partition that is
associated to a clock has to be categorized as ``\textbf{clocked
discrete-time}'' or ``\textbf{clocked discretized continuous-time}''
partition.

If a clocked partition contains no operator \textbf{der},
\textbf{delay}, \textbf{spatialDistribution}, no event related operators
from \autoref{event-related-operators-with-function-syntax} (with exception of \textbf{noEvent}(..)), and no
\textbf{when}-clause with a Boolean condition, it is a ``\textbf{clocked
discrete-time}'' partition {[}\emph{that is, it is a standard sampled
data system that is described by difference equations.}{]}

If a clocked partition is not a ``clocked discrete-time'' partition, it
is a ``\textbf{clocked discretized continuous-time}'' partition. Such a
partition has to be solved with a ``solver method'' of \autoref{solver-methods}.
When previous(x) is used on a continuous-time state variable x, then
previous(x) uses the start value of x as value for the first clock tick.

In a clocked discrete-time partition all event generating mechanisms do
no longer apply. Especially neither relations, nor one of the built-in
operators of \autoref{event-triggering-mathematical-functions} (event triggering mathematical functions)
will trigger an event.

\subsection{Solver Methods}\doublelabel{solver-methods}

The integration method associated with a clocked discretized
continuous-time partition is defined with a string. A predefined type
ModelicaServices.Types.SolverMethod defines the methods supported by the
respective tool by using the choices annotation. {[}\emph{The
ModelicaServices package contains tool specific definitions. A string is
used instead of an enumeration, since different tools might have
different values and then the integer mapping of an enumeration is
misleading since the same value might characterize different
integrators.}{]} The following names of solver methods are standardized:

\begin{lstlisting}[language=modelica]
type SolverMethod = String annotation(choices(
  choice="External" "Solver specified externally",
  choice="ExplicitEuler" "Explicit Euler method (order 1)",
  choice="ExplicitMidPoint2" "Explicit mid point rule (order 2)",
  choice="ExplicitRungeKutta4" "Explicit Runge-Kutta method (order 4)",
  choice="ImplicitEuler" "Implicit Euler method (order 1)",
  choice="ImplicitTrapezoid" "Implicit trapezoid rule (order 2)"
)) "Type of integration method to solve differential equations in a clocked discretized"
   +"continuous-time partition."
\end{lstlisting}

If a tool supports one of the integrators of SolverMethod, it must use
the solver method name of above. {[}\emph{A tool may support also other
integrators. Typically, a tool supports at least methods ``External''
and ``ExplicitEuler''. If a tool does not support the integration method
defined in a model, typically a warning message is printed and the
method is changed to ``External''.}{]}

If the solver method is "External", then the partition associated with
this method is integrated by the simulation environment for an interval
of length of interval() using a solution method defined in the
simulation environment {[}\emph{(for example by having a table of the
clocks that are associated with discretized continuous-time partitions
and a method selection per clock). In such a case, the solution method
might be a variable step solver with step-size control that integrates
between two clock ticks. The simulation environment might also combine
all partitions associated with method "External", as well as all
continuous-time partitions, and integrate them together with the solver
selected by the simulation environment.}{]}

If the solver method is \textbf{not} "External", then the partition is
integrated using the given method with the step-size interval().
{[}\emph{For a periodic clock, the integration is thus performed with
fixed step size.}{]}

The solvers are defined with respect to the underlying ordinary
differential equation in state space form to which the continuous-time
partition can be transformed, at least conceptually (\emph{t} is time,
\textbf{u}\textsubscript{c}(\emph{t}) is the continuous-time Real vector
of input variables, \textbf{u}\textsubscript{d}(\emph{t}) is the
discrete-time Real/Integer/Boolean/String vector of input variables,
\textbf{x}(\emph{t}) is the continuous-time real vector of states, and
\textbf{y}(\emph{t}) is the continuous-time or discrete-time
Real/Integer/Boolean/String vector of algebraic and/or output
variables):
\begin{eqnarray*}
\dot{x}&=&f(x, u, t)\\
y&=&g(x, u, t)
\end{eqnarray*}
A solver method is applied on a subclock partition. Such a partition has
explicit inputs \textbf{u} marked by \textbf{sample}(u),
\textbf{subSample}(u), \textbf{superSample}(u), \textbf{shiftSample}(u)
and/or \textbf{backSample}(u). Furthermore, the outputs \textbf{y} of
such a partition are marked by \textbf{hold}(y), \textbf{subSample}(y),
\textbf{superSample}(y), \textbf{shiftSample}(y), and/or
\textbf{backSample}(y). The arguments of these operators are to be used
as input signals \textbf{u} and output signals \textbf{y} in the
conceptual ordinary differential equation above, and in the
discretization formulae below, respectively.

The solver methods (with exception of "External") are defined by
integrating from clock tick \emph{t}\textsubscript{i-1} to clock tick
\emph{t}\textsubscript{i} and computing the desired variables at
\emph{t}\textsubscript{i}, with \emph{h} = \emph{t}\textsubscript{i} --
\emph{t}\textsubscript{i-1} = interval(u) and
\textbf{x}\emph{\textsubscript{i}} =
\textbf{x}(\emph{t\textsubscript{i}}):

\begin{longtable}[]{|p{3.5cm}|p{11cm}|}
\hline
\emph{SolverMethod} &
\emph{Solution method}

(for all methods: $y_i=g(x_i,u_{c,i},u_{d,i},t_i)$)\\ \hline
\endhead
ExplicitEuler &
$\begin{aligned}
x_{i} &:= x_{i-1}+h\cdot\dot{x}_{i-1}\\
\dot{x}_{i} &:= f(x_i,u_{c,i},u_{d,i},t_i)
\end{aligned}$\\ \hline
ExplicitMidPoint2 &
$\begin{aligned}
x_{i} &:= x_{i-1}+h\cdot f(x_{i-1}+\frac{1}{2}\cdot h \cdot\dot{x}_{i-1},\frac{u_{c,i-1}+u_{c,i}}{2},u_{d,i-1},t_{i-1}+\tfrac{1}{2}\cdot h)\\
\dot{x}_{i} &:= f(x_i,u_{c,i},u_{d,i},t_i)
\end{aligned}$\\ \hline
ExplicitRungeKutta4 &
$\begin{aligned}
k_1 &:= h\cdot \dot{x}_{i-1}\\
k_2 &:= h\cdot f(x_{i-1}+\tfrac{1}{2}k_1,\frac{u_{c,i-1}+u_{c,i}}{2},u_{d,i-1},t_{i-1}+\tfrac{1}{2}\cdot h)\\
k_3 &:= h\cdot f(x_{i-1}+\tfrac{1}{2}k_2,\frac{u_{c,i-1}+u_{c,i}}{2},u_{d,i-1},t_{i-1}+\tfrac{1}{2}\cdot h)\\
k_4 &:= h\cdot f(x_{i-1}+k_3,u_{c,i},u_{d,i},t_i)\\
x_{i} &:= x_{i-1}+\tfrac{1}{6}\cdot(k_1+2\cdot k_2+2\cdot k_3+k_4)\\
\dot{x}_{i} &:= f(x_i,u_{c,i},u_{d,i},t_i)
\end{aligned}$
\\ \hline
ImplicitEuler &$\begin{aligned}
x_{i} &= x_{i-1}+h\cdot\dot{x}_i \textrm{// equations system with unknowns:} x_i,\dot{x}_i\\
\dot{x}_{i} &= f(x_i,u_{c,i},u_{d,i},t_i)
\end{aligned}$\\ \hline
ImplicitTrapezoid &$\begin{aligned}
x_{i} &= x_{i-1}+\tfrac{1}{2}h\cdot(\dot{x}_i+\dot{x}_{i-1}) \textrm{// equations system with unknowns:} x_i,\dot{x}_i\\
\dot{x}_{i} &= f(x_i,u_{c,i},u_{d,i},t_i)
\end{aligned}$\\ \hline
\end{longtable}

The initial conditions will be used at the first tick of the clock, and
the first integration step will go from the first to the second tick of
the clock.

{[}\emph{Example:}
\emph{Assume the differential equation}
\begin{lstlisting}[language=modelica]
  input Real u;
  Real x(start=1, fixed=true);
equation
  der(x) = -x + u
\end{lstlisting}
\emph{shall be transformed to a clocked discretized continuous-time
partition with the ExplicitEuler method. The following model is a manual
implementation:}

\begin{lstlisting}[language=modelica]
  input Real u;
  parameter Real x_start = 1;
  Real x(start=x_start); // previous(x) = x_start at first clock tick
  Real der_x(start=0); // previous(der_x) = 0 at first clock tick
protected
  Boolean first(start=true);
equation
  when Clock() then
    first = false;
    if previous(first) then
      // first clock tick (initialize system)
      x = previous (x);
    else
      // second and further clock tick
      x = previous (x) +
      interval()*previous(der_x);
    end if;
    der_x = -x + u;
  end when;
\end{lstlisting}
{]}

{[}\emph{For the implicit integration methods the efficiency can be
enhanced by utilizing the discretization formula during the symbolic
transformation of the equations. For example, linear differential
equations are then mapped to linear and not non-linear algebraic
equation systems, and also the structure of the equations can be
utilized. For details see (Elmqvist et. al. 1995). It might be necessary
to associate additional data for an implicit integration method, e.g.
the relative tolerance to solve the non-linear algebraic equation
systems, or the maximum number of iterations in case of hard realtime
requirements. This data is tool specific and is typically either defined
with a vendor annotation or is given in the simulation environment.}{]}

\subsection{Associating a Solver to a Partition}\doublelabel{associating-a-solver-to-a-partition}

A solverMethod can be associated to a clock with the overloaded Clock
constructor Clock(c, solverMethod), see \autoref{clock-constructors}. If a clock is
associated with a clocked partition and a solverMethod is associated
with this clock, then the partition is integrated with it.

{[}\emph{Example:}
\begin{lstlisting}[language=modelica]
  // Continuous PI controller in a clocked partition
  vd = sample(x2, Clock(Clock(1,10),solverMethod="ImplicitEuler"));
  e = ref-vd;
  der(xd) = e/Ti;
  u = k*(e + xd);

  // Physical model
  f = hold(u);
  der(x1) = x2;
  m*der(x2) = f;
\end{lstlisting}
{]}

\subsection{Inferencing of solverMethod}\doublelabel{inferencing-of-solvermethod}

If a solverMethod is not explicitly associated with a partition, it is
inferred with a similar mechanism as for sub-clock inferencing, see
\autoref{sub-clock-inferencing}. The inferencing mechanism is defined using the operator
``solverExplicitlyDefined(c)'' which returns \textbf{true}, if a
solverMethod is explicitly associated with clock c and returns
\textbf{false} otherwise.

For every partitioning operator of \autoref{partitioning-operators}, two clocks c1 and c2
are defined for the input and the output argument of the operator,
respectively. Furthermore, for every equality and assignment of clocks,
c1 = c2 or c1 := c2, two clocks are defined as well. In all these cases,
the following statements are implicitly introduced:

\begin{lstlisting}[language=modelica]
if solverExplicitlyDefined(c1) and solverExplicitlyDefined(c2) then
  // o.k. (no action)
elseif not solverExplicitlyDefined(c1) and not solverExplicitlyDefined(c2) then
  assert(c1.solverMethod == c2.solverMethod);
elseif solverExplicitlyDefined(c1) <>solverExplicitlyDefined(c2) then
  c1.solverMethod = c2.solverMethod
end if;
\end{lstlisting}

The introduced set of, potentially underdetermined or overdetermined
constraints has to be solved. If no solution exists or if a solution is
contradictory on some clocks that are associated with clocked
discrete-time partitions, then this is ignored, since no solverMethod is
needed for such partitions.

{[}\emph{Example:}

\begin{lstlisting}[language=modelica]
model InferenceTest "Specific clocks set on all partitions. Decouple constraint"
  ...
equation
  // Physical model
  der(x1) = -100*x1 + hold(z2);
  // Controller submodels
  der(z1) = -100*z1 + sample(x2,
  Clock(Clock(1, 100), solverMethod="ImplicitEuler"));
  w = 0.9*previous (w) + superSample(z1, 3);
  when Clock(Clock(1,20),
    solverMethod="ExplicitEuler") then
    ww = superSample (w);
  end when
  der(z2) = -z2 + ww;
end InferenceTest;
\end{lstlisting}
{]}

\section{Initialization of Clocked Partitions}\doublelabel{initialization-of-clocked-partitions}

The standard scheme for initialization of Modelica models does not apply
for clocked discrete-time partitions. Instead, initialization is
performed in the following way:

\begin{itemize}
\item
  Clocked discrete-time variables cannot be used in initial equation or
  initial algorithm sections.
\item
  Attribute ``fixed'' cannot be applied on clocked discrete-time
  variables. The attribute ``fixed'' is true for variables to which the
  \textbf{previous} operator is applied, otherwise false.
\end{itemize}

\section{Other Operators}\doublelabel{other-operators}

The following additional utility operators are provided:

\begin{longtable}[]{|l|p{12cm}|}
\hline \endhead
\textbf{firstTick}(u)&
This operator returns true at the first tick of the clock of the
expression, in which this operator is called. The operator returns false
at all subsequent ticks of the clock. The optional argument u is only
used for clock inference, see \autoref{clock-partitioning}.\\ \hline
\textbf{interval}(u)&
This operator returns the interval between the previous and present tick
of the clock of the expression, in which this operator is called. The
optional argument u is only used for clock inference, see \autoref{clock-partitioning}.
At the first tick of the clock the following is returned: a) if the
specified clock interval is parametric, this value is returned; b)
otherwise the start value of the variable specifying the interval is
returned; c) for an event clock the additional startInterval argument to
the event clock constructor is returned. The return value of the
interval operator is a scalar Real number.
\\ \hline
\end{longtable}

It is an error if these operators are called in the continuous-time
partition.

\emph{{[}Example:}

\emph{A discrete PI controller is parameterized with the parameters of a
continuous PI controller, in order that the discrete block is robust
against changes in the sample period. This is achieved by discretizing a
continuous PI controller (here with an implicit Euler method):}

\begin{lstlisting}[language=modelica]
block ClockedPI
  parameter Real T "Time constant of continuous PI controller";
  parameter Real k "Gain of continuous PI controller";
  input Real u;
  output Real y;
  Real x(start=0);
  protected
  Real Ts = interval(u);
equation
  /* Continuous PI equations: der(x) = u/T; y = k*(x + u);
     Discretization equation: der(x) = (x - previous (x))/Ts;
  */
  when Clock() then
    x = previous (x) + Ts/T*u;
    y = k*(x + u);
  end when;
end ClockedPI;
\end{lstlisting}
\emph{A continuous-time model is inverted, discretized and used as
feedforward controller for a PI controller\\
(der(..), previous, interval are used in the same partition):}

\begin{lstlisting}[language=modelica]
block MixedController
  parameter Real T "Time constant of continuous PI controller";
  parameter Real k "Gain of continuous PI controller";
  input Real y_ref, y_meas;
  Real y;
  output Real yc;
  Real z(start=0);
  Real xc(start=1, fixed=true);
  Clock c = Clock(Clock(0.1), solverMethod="ImplicitEuler");
protected
  Real uc;
  Real Ts = interval(uc);
equation
  /* Continuous-time, inverse model */
  uc = sample(y_ref, c);
  der(xc) = uc;
  /* PI controller */
  z = if  firstTick() then 0 else
  previous(z) + Ts/T*(uc - y_meas);
  y = xc + k*(xc + uc);
  yc = hold (y);
end MixedController;
\end{lstlisting}
\emph{{]}}

\section{Semantics}\doublelabel{semantics}

The execution of sub partitions requires exact time management for
proper synchronization. The implication is that testing a Real valued
time variable to determine sampling instants is not possible. One
possible method is to use counters to handle sub-sampling scheduling.

\begin{lstlisting}[language=modelica]
Clock_i_j_ticks = if pre(Clock_i_j_ticks)<subSamplingFactor_i_j then 1+pre(Clock_i_j_ticks) else 1;
\end{lstlisting}

and to test the counter to determine when the sub-clock is ticking:
\begin{lstlisting}[language=modelica]
Clock_i_j_activated = BaseClock_i_activated and Clock_i_j_ticks >= subSamplingFactor_i_j;
\end{lstlisting}
The Clock\_i\_j\_activated flag is used as the guard for the sub
partition equations.

\emph{{[} Consider the following example:}

\begin{lstlisting}[language=modelica]
model ClockTicks
  Integer second = sample(1, Clock(1));
  Integer seconds(start=-1) = mod(previous(seconds) + second, 60);
  Integer milliSeconds(start=-1)=
      mod(previous(milliSeconds) + superSample(second, 1000), 1000);
  Integer minutes(start=-1)=
      mod(previous(minutes) + subSample(second, 60), 60);
end ClockTicks;
\end{lstlisting}

\emph{A possible implementation model is shown below using Modelica 3.2
semantics. The base-clock is determined to 0.001 seconds and the
sub-sampling factors to 1000 and 60000.}

\begin{lstlisting}[language=modelica]
model ClockTicksWithModelica32
   Integer second;
   Integer seconds(start = -1);
   Integer milliSeconds(start = -1);
   Integer minutes(start = -1);

   Boolean BaseClock_1_activated;
   Integer Clock_1_1_ticks(start=59999);
   Integer Clock_1_2_ticks(start=0);
   Integer Clock_1_3_ticks(start=999);
   Boolean Clock_1_1_activated;
   Boolean Clock_1_2_activated;
   Boolean Clock_1_3_activated;
equation
  // Prepare clock tick
  BaseClock_1_activated =  sample(0, 0.001);
  when BaseClock_1_activated then
     Clock_1_1_ticks = if pre(Clock_1_1_ticks) < 60000 then 1+pre(Clock_1_1_ticks) else 1;
     Clock_1_2_ticks = if pre(Clock_1_2_ticks) < 1 then 1+pre(Clock_1_2_ticks) else 1;
     Clock_1_3_ticks = if pre(Clock_1_3_ticks) < 1000 then 1+pre(Clock_1_3_ticks) else 1;
  end when;
  Clock_1_1_activated =  BaseClock_1_activated and Clock_1_1_ticks >= 60000;
  Clock_1_2_activated =  BaseClock_1_activated and Clock_1_2_ticks >= 1;
  Clock_1_3_activated =  BaseClock_1_activated and Clock_1_3_ticks >= 1000;

  // -----------------------------------------------------------------------------
  // Sub partition execution
  when {Clock_1_3_activated} then
     second = 1;
  end when;
  when {Clock_1_1_activated} then
      minutes = mod(pre(minutes)+second, 60);
  end when;
  when {Clock_1_2_activated} then
     milliSeconds = mod(pre(milliSeconds)+second, 1000);
  end when;
  when {Clock_1_3_activated} then
     seconds = mod(pre(seconds)+second, 60);
  end when;
end ClockTicksWithModelica32;
\end{lstlisting}
\emph{{]}}

\end{document}
